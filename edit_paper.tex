\documentclass[conference]{IEEEtran}
\usepackage[utf8]{inputenc}

%\usepackage[portuguese]{babel}
%
%\usepackage{glossaries}  
%\setacronymstyle{short-long}
%
%\newacronym{mpls}{MPLS}{Multiprotocol Label Switching}
%\newacronym{asic}{ASIC}{Application-specific Integrated Circuit}
%
%\makeglossaries

\usepackage{multirow}

\usepackage[pdftex]{graphicx}
\graphicspath{{./Imagens/}}

% correct bad hyphenation here
\hyphenation{op-tical net-works semi-conduc-tor DEFCON}
%

\begin{document}
\title{How private is my medical information?\\
  \large Sociologia e Ética da Informática\\
  2017/2018				%%Colocar data
}

\author{
\IEEEauthorblockN{Vanessa Silva}
\IEEEauthorblockA{Faculdade de Ciências da Universidade do Porto\\Email: up201305731@fc.up.pt}
}

\maketitle

\begin{abstract}

% deve indicar de forma clara o que vai tratar o relatório (5-6 linhas).

\end{abstract}


\IEEEpeerreviewmaketitle


\section{Introdução}

% introduzir o tema de forma clara (o que é, em que consiste, quando e como surgiu, para que serve, ...), fazer o enquadramento (semelhanças e/ou diferenças) com os restantes tópicos apresentados nas aulas e descrever sumariamente a organização do relatório.

\section{Secções principais sobre o tema}

% organizar segundo o tema em questão e segundo os tópicos que pretende realçar. Uma das secções pode ser usada para referenciar ou comparar trabalhos relacionados.

\section{Conclusões}

% resumir o que foi apresentado e indicar possíveis tópicos de continuação ou perspectivas de futuro/sucesso (10-15 linhas).

\bibliography{ref}{}
\bibliographystyle{IEEEtran}

\end{document}
